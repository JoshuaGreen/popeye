\documentclass{article}%
\usepackage{diagram}
\usepackage{german}

\begin{document}

\begin{diagram}%
 \author{Kuhlmann, J"org; Sch"oneberg, Achim}%
 \sourcenr{4216}%
 \year{1984}%
 \month{11}%
 \issue{71}%
 \source{feenschach}%
 \pieces{wKd5, wTa6, sSa8, sKb8}%
 \stipulation{h\#4}%
 \condition{Circe}%
 \remark{Beamtet a6 a8 b8
}%
 \solution{%
  1.bSa8-c7 bTa6-b6   2.bKb8-a8 Kd5-c6   3.bSc7-d5 bTb6-a6   4.bSd5-b6 Kc6*b6 [+sbSb8] \# 
Loesung beendet. 
 }%
 \Co+%
\end{diagram}
\hfill
\begin{diagram}%
 \author{Kuhlmann, J"org; Sch"oneberg, Achim}%
 \sourcenr{i}%
 \year{1984}%
 \month{11}%
 \issue{71}%
 \source{feenschach}%
 \pieces{sKf1, wKf3, sDa6, wTf6}%
 \stipulation{h\#2}%
 \remark{Beamtet f3 a6 f6
}%
 \solution{%
  1.bDa6-a1 bTf6-a6   2.bDa1-f6 + bTa6-a1 \# 
Loesung beendet. 
 }%
 \Co+%
\end{diagram}
\hfill
\begin{diagram}%
 \author{Sch"oneberg, Achim}%
 \sourcenr{ii}%
 \year{1984}%
 \month{11}%
 \issue{71}%
 \source{feenschach}%
 \pieces{wBb6, sBa7, sKa8, wKc8}%
 \stipulation{h=2}%
 \remark{Beamtet a7 c8
}%
 \solution{%
  1.bBa7-a6 b6-b7 +   2.Ka8-a7 b7-b8=T = 
Loesung beendet. 
 }%
 \Co+%
\end{diagram}
\hfill
\begin{diagram}%
 \author{Kuhlmann, J"org}%
 \sourcenr{iv}%
 \year{1984}%
 \month{11}%
 \issue{71}%
 \source{feenschach}%
\dedication{G"unter B"using gewidmet
}%
 \pieces{sKa1, sSd1, sSb3, wTd3}%
 \stipulation{h\#3}%
 \condition{BeamtenSchach}%
 \solution{%
  1.Sd1-b2 Td3-f3   2.Sb3-d2 Tf3-d3   3.Sd2-b1 Td3-a3 \# 
  1.Sb3-c1 Td3-c3   2.Sc1-a2 Tc3-d3   3.Sd1-b2 Td3-d1 \# 
Loesung beendet. 
 }%
 \Co+%
\end{diagram}
\hfill
\begin{diagram}%
 \author{Kuhlmann, J"org; Sch"oneberg, Achim}%
 \sourcenr{v}%
 \year{1984}%
 \month{11}%
 \issue{71}%
 \source{feenschach}%
 \pieces{sBe2, wLc3, wLd3, wBb4, wSa5, sKb5, sBa6, sTb6, sBc6, wBe7, wKh8}%
 \stipulation{h\#2}%
 \remark{Beamtet e2 d3 a5 b6 e7
}%
 \twins{b) \wS a5{\ra}c5}%
 \solution{%
a)  1.bBe2-e1=bT bBe7-e8=bL   2.bTe1-e5 bLe8*c6 \# 
b)  1.bBe2-e1=bD bBe7-e8=bS   2.bDe1-e7 bSe8-c7 \# 
Loesung beendet. 
 }%
 \Co+%
\end{diagram}
\hfill
\begin{diagram}%
 \author{Wedekind, Claus}%
 \sourcenr{234}%
 \year{1995}%
 \month{12}%
 \day{16}%
 \issue{45}%
 \source{harmonie}%
 \pieces{sDa1, wLg1, wLh1, sBa2, wSb2, sLd2, sLe2, sSa3, sSc3, wBf3, wBg3, sBd5, sKe5, wSg5, wTa6, wTf6, sTe7, wBf7, wKg7, wDc8}%
 \stipulation{\#2}%
 \condition{BeamtenSchach}%
 \solution{%
   1.Lg1-a7 ! Drohung:
          2.f3-f4 \#
      1...Sa3-b1
          2.Sb2-c4 \#
      1...Sa3-c4
          2.Sb2*c4 \#
      1...Sa3-b5
          2.Sb2-c4 \#
      1...Sc3-b1
          2.Sb2-d3 \#
      1...Sc3-d1
          2.Sb2-d3 \#
      1...Sc3-b5
          2.Sb2-d3 \#
Loesung beendet. 
 }%
 \Co+%
\end{diagram}
\hfill
\begin{diagram}%
 \author{Kuhlmann, J"org; Sch"oneberg, Achim}%
 \sourcenr{vi}%
 \year{1984}%
 \month{11}%
 \issue{71}%
 \source{feenschach}%
\dedication{Verbesserung 3/1999
}%
 \pieces{wKc2, wLg2, wLe3, sSg4, sBa5, wTe5, sKa6, wBb6, wBb7, sLb8}%
 \stipulation{\#7}%
 \condition{Circe}%
 \remark{Beamtet c2 e3 g4 a5 e5 a6 b6 b7 b8
}%
 \solution{%
   1.Lg2-c6 !
      1...bBa5-a4
          2.Lc6*a4 [+sbBa7]
              2...bBa7*b6 [+wbBb2]
                  3.La4-b3
                      3...bBb6-b5
                          4.Lb3-c4
                              4...bBb5*c4 [+wLf1]
                                  5.bTe5-b5
                                      5...bBc4-c3
                                          6.Lf1-g2
                                              6...bBc3*b2
                                                  7.bTb5-a5 \#
Loesung beendet. 
 }%
\end{diagram}
\hfill
\begin{diagram}%
 \author{Kuhlmann, J"org; Sch"oneberg, Achim}%
 \sourcenr{iii}%
 \year{1984}%
 \month{11}%
 \issue{71}%
 \source{feenschach}%
\dedication{G"unter B"using
Verbesserung Europa Rochade Wann?
}%
 \pieces{wTa1, sSa3, sBg7, sKh8}%
 \stipulation{h\#11}%
 \condition{BeamtenSchach}%
 \solution{%
  1.Sa3-c2 Ta1-a2   2.Sc2-b4 Ta2-b2   3.Sb4-d3 Tb2-b3   4.Sd3-c5 Tb3-c3   5.Sc5-e4 Tc3-c4   6.Se4-d6 Tc4-d4   7.Sd6-f5 Td4-d5   8.Sf5-e7 Td5-e5   9.Se7-g6 Te5-e6  10.Sg6-f8 Te6-f6  11.Sf8-h7 Tf6-f8 \# 
Loesung beendet. 
 }%
 \Co+%
\end{diagram}
\hfill

\putsol

\end{document}
